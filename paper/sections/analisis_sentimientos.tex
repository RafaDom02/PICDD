


\subsubsection{Técnicas de análisis de sentimiento aplicadas al fútbol.}

Las plataformas de redes sociales, con especial énfasis en Twitter, se han consolidado como la fuente principal para la extracción del pulso emocional de la afición deportiva. Millones de tweets generados durante eventos deportivos o en las fases previas a estos sirven como un corpus masivo para capturar la opinión y el sentimiento de los seguidores. A modo de ejemplo, Wunderlich y Memmert (2021) recopilaron aproximadamente dos millones de tweets correspondientes a partidos de la Premier League con el objetivo de analizar el sentimiento del público en tiempo real \cite{Wunderlich2022}. Sobre este tipo de datos se aplican técnicas de preprocesamiento, seguidas de algoritmos de clasificación de sentimiento. Inicialmente, diversos trabajos emplearon enfoques basados en lexicones o herramientas de propósito general (como OpinionFinder para el idioma inglés) con el fin de medir la polaridad de los mensajes \cite{Schumaker2016}. No obstante, la tendencia actual se ha orientado hacia la adaptación de modelos de aprendizaje automático supervisado (tales como Support Vector Machines (SVM), Naïve Bayes, Random Forest) e incluso arquitecturas de aprendizaje profundo, al dominio específico del fútbol.\\

Un desafío recurrente y bien identificado en esta área es la divergencia inherente entre el lenguaje deportivo y el lenguaje común. Por esta razón, una línea de investigación se ha centrado en el desarrollo de lexicones específicos para el ámbito futbolístico o en el ajuste y la optimización de modelos para dicho dominio. En este sentido, Aloufi y El Saddik (2018) realizaron un estudio comparativo de distintos algoritmos de clasificación de texto (SVM, Naïve Bayes multinomial y Random Forest) y diversos métodos de representación de características (Bolsa de Palabras – Bag-of-Words– versus etiquetado gramatical POS, y el uso de lexicones). Dicho estudio se basó en un conjunto de datos de tweets relativos a la UEFA Champions League y la Copa Mundial de la FIFA 2014 \cite{Selak2024}. Los autores encontraron que un modelo SVM, utilizando representaciones de Bolsa de Palabras, superaba a las demás combinaciones evaluadas en términos de precisión \cite{Selak2024}. En una investigación posterior, Aloufi et al. (2018) construyeron un léxico de sentimiento específico para el fútbol, el cual integraba vocabulario característico de la afición, y con su incorporación en los modelos lograron una mejora en la detección de sentimiento de aproximadamente un 6\% \cite{Selak2024}. Adicionalmente, observaron que los modelos basados en unigramas (palabras individuales) ofrecían un rendimiento superior a aquellos basados en bigramas para este tipo de análisis \cite{Selak2024}. Estas mejoras acumuladas sugieren de manera consistente la importancia fundamental de adaptar las técnicas de Procesamiento del Lenguaje Natural (NLP) al contexto deportivo, lo que implica considerar el vocabulario futbolístico específico, las formas de expresión típicas de los aficionados e, incluso, las particularidades idiomáticas locales del seguidor.\\

Si bien Twitter ha predominado como objeto de estudio debido a su vasto volumen de datos y accesibilidad, el análisis del sentimiento en el contexto deportivo se ha extendido a otras fuentes de opinión pública. Entre estas se incluyen foros de aficionados, secciones de comentarios en portales de noticias deportivas y, de manera más incipiente, el contenido generado durante las retransmisiones en vivo. A modo ilustrativo, se encuentran en desarrollo sistemas capaces de procesar en tiempo real los comentarios emitidos por los espectadores durante las transmisiones de partidos, llegando incluso a traducir las emociones detectadas a representaciones icónicas (emojis) como forma de sumarizar la reacción colectiva del público \cite{Ortu2024}. En el caso de los datos extraídos de foros o secciones de comentarios en noticias, las metodologías de Procesamiento del Lenguaje Natural aplicadas –tales como el preprocesamiento y el análisis de sentimiento– son análogas a las utilizadas con los datos de Twitter. No obstante, estas técnicas requieren ajustes para adecuarse a las particularidades del registro lingüístico de dichas plataformas, que suelen caracterizarse por un menor empleo de sociolectos digitales (comúnmente referidos como "jerga de internet") y una mayor extensión y estructura en los mensajes. Independientemente de la fuente de datos, el propósito fundamental del análisis de sentimiento en el dominio deportivo es la cuantificación del estado anímico o la atmósfera emocional que circunda a un equipo o jugador determinado. Esto abarca tanto las fases previas a un evento deportivo, donde se manifiestan expectativas, niveles de optimismo o preocupación, como las fases posteriores al mismo, caracterizadas por reacciones que pueden oscilar entre la euforia, la crítica constructiva o destructiva, y la decepción, entre otras.

\subsubsection{Sentimiento de la afición y modelos de rendimiento deportivo.}

Diversos estudios en el ámbito del fútbol profesional han explorado la hipótesis de que el estado anímico de la afición, cuantificado mediante análisis de sentimiento, posee la capacidad de influir o predecir el rendimiento de los equipos en el terreno de juego. En términos generales, la evidencia empírica sugiere que el sentimiento colectivo de los seguidores proporciona señales informativas, aunque no concluyentes, respecto al resultado final, y su influencia tiende a ser limitada o de naturaleza indirecta. No obstante, su integración en modelos cuantitativos ha evidenciado mejoras marginales en determinados escenarios y ha desvelado aspectos relevantes sobre la intrincada dinámica entre la afición y el desempeño deportivo.\\

En el ámbito de la predicción de resultados, Schumaker et al. (2016) realizaron uno de los estudios seminales al combinar el análisis de sentimiento en Twitter con la predicción de partidos de la Premier League. Mediante la recopilación de tweets concernientes a cada club antes de los encuentros, desarrollaron modelos predictivos de resultados (victoria/derrota) basados en el sentimiento agregado de la afición, los cuales fueron contrastados con enfoques convencionales fundamentados en las cuotas de apuestas (favoritos según las casas de apuestas) \cite{Schumaker2016}. Sus resultados indicaron que los modelos basados en cuotas de apuestas mantenían una ligera superioridad en precisión frente a aquellos basados únicamente en sentimiento; sin embargo, resulta destacable que diversos modelos basados en sentimiento generaron una rentabilidad superior en simulaciones de apuestas en comparación con la estrategia del favorito, lo que sugiere la captura de información no completamente incorporada en las probabilidades establecidas \cite{Selak2024}. De manera análoga, Godin et al. (2014) integraron múltiples señales provenientes de redes sociales —tales como el volumen de tweets, el sentimiento promedio e incluso predicciones explícitas de usuarios— con el objetivo de superar el rendimiento de los pronósticos ofrecidos por las casas de apuestas. Al probar su metodología en aproximadamente 200 partidos de la Premier League, reportaron una precisión aproximada del 52\% para un clasificador basado exclusivamente en sentimiento, mientras que la fusión de este con otras variables elevó la precisión al 68\%, superando a las cuotas de apuestas en determinados escenarios \cite{Selak2024}. Estos trabajos indican que el sentimiento público constituye una señal independiente que, al ser integrada con datos históricos o probabilísticos, posee el potencial de optimizar marginalmente las predicciones de resultados deportivos.\\

Más allá de la predicción de resultados, el análisis de sentimiento se ha empleado para dilucidar la relación existente entre la atmósfera emocional que rodea a un equipo y su rendimiento efectivo en el campo. Un estudio reciente de Ortu et al. (2024) analizó 167.841 tweets referentes a jugadores de la English Premier League (EPL) para explorar esta interrelación antes y después de los partidos \cite{Ortu2024}. Sus hallazgos revelan correlaciones significativas; por ejemplo, un sentimiento negativo predominante entre los aficionados previo al encuentro demostró ser un indicador predictivo más robusto del desempeño subsecuente (medido a través de estadísticas individuales) en comparación con el sentimiento positivo \cite{Ortu2024}. En otras palabras, la manifestación de pesimismo o críticas por parte de los aficionados hacia jugadores específicos antes del partido se correlacionó significativamente con el rendimiento de dichos jugadores durante el encuentro subsiguiente, en mayor medida que el mero optimismo expresado por los seguidores \cite{Ortu2024}. Posteriormente al partido, esta relación se modificaba: las reacciones emocionales de la afición se alineaban con el resultado obtenido, reflejando la respuesta del público al desempeño observado (expresiones de elogio si el equipo o jugador superaba las expectativas, o una mayor negatividad en caso de decepción) \cite{Ortu2024}. Este estudio sugiere que el denominado "clima emocional" previo —particularmente las expresiones de duda o crítica— podría ejercer una presión o motivación adicional sobre los jugadores, afectando su rendimiento de manera cuantificable.\\

Otras investigaciones han procurado cuantificar de manera agregada la moral del equipo mediante el análisis de estas señales sociales. Por ejemplo, Selak (2024) analizó los sentimientos expresados en Twitter hacia clubes de primera división antes de los partidos y los comparó con la actuación posterior del equipo. Sus hallazgos indicaron que un clima predominantemente positivo entre la afición previo al encuentro se asociaba con un incremento marginal (aproximadamente 1\%) en la probabilidad de victoria, concluyendo que el sentimiento pre-partido de los seguidores no ejerce un impacto significativo en el rendimiento del equipo \cite{Selak2024}. En una línea similar, Wunderlich y Memmert (2021) exploraron si el sentimiento manifestado en tiempo real durante los partidos podría contribuir a la predicción de eventos específicos, como la consecución de goles. Utilizando algoritmos de aprendizaje automático (regresión logística, Random Forest) con datos de tweets recolectados minuto a minuto, constataron que estos modelos en tiempo real no lograban superar la precisión de las predicciones fundamentadas en información pre-partido (por ejemplo, las probabilidades iniciales de apuesta) \cite{Selak2024}. Ello sugiere que, durante el transcurso del juego, variables como el marcador y las estadísticas convencionales ejercen una influencia preponderante sobre las fluctuaciones inmediatas del estado anímico en redes sociales, las cuales, adicionalmente, tienden a manifestar una mayor negatividad conforme avanza un encuentro disputado y tenso \cite{Selak2024}.\\

A pesar de la heterogeneidad en los resultados, existe un consenso generalizado en la literatura respecto al valor añadido que las señales de sentimiento aportan a la comprensión de la dinámica entre el equipo y su afición. La incorporación del estado anímico de la afición, considerado como un proxy de la "moral" en el entorno del equipo, permite una contextualización más rica del rendimiento; por ejemplo, permitiendo discernir si un declive en el rendimiento deportivo se correlaciona con un incremento de las críticas en redes sociales, o si una racha de resultados positivos se acompaña de un entusiasmo generalizado entre los seguidores. Algunos sistemas de Big Data aplicados al deporte ya integran estas métricas de sentimiento con datos de juego. Souza et al. (2021) propusieron un sistema integral de predicción para el fútbol que no solo incorporaba datos numéricos de los partidos, sino también el sentimiento extraído de tweets e incluso el análisis de las expresiones faciales de los jugadores \cite{Selak2024}. En su propuesta, el análisis de sentimiento facilitó la identificación de las emociones predominantes de los seguidores a lo largo del encuentro, proveyendo información de considerable valor para cuerpos técnicos y directivos en la comprensión de la respuesta de la afición y, consecuentemente, en la posible adopción de medidas orientadas a fortalecer la conexión entre el equipo y sus seguidores \cite{Selak2024}. En última instancia, estas aplicaciones ilustran la función del análisis de sentimiento como un "termómetro" del ambiente que rodea a un equipo: aunque no determina per se los resultados deportivos, sí aporta un componente socio-psicológico cuantificable que complementa las estadísticas tradicionales en el análisis del rendimiento deportivo de los equipos de fútbol profesional \cite{Selak2024}.
