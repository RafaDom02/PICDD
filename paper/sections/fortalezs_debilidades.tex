La predicción de resultados en el fútbol profesional ha empleado significativamente la evaluación de las capacidades de los equipos y jugadores, esto lo podemos ver utilizado desde papers hasta en aplicaciones móviles de resultados deportivos. Un componente central de esta evaluación es la identificación y cuantificación de sus fortalezas y debilidades, así como las de sus jugadores.

\subsubsection{Medición de la Fortaleza General del Equipo Mediante Sistemas de Calificación}

Una aproximación fundamental para medir la capacidad competitiva de los equipos se basa en sistemas de calificación (rating systems) que buscan condensar su fortaleza general en una métrica única o un conjunto reducido de ellas.

\begin{itemize}

    \item Métodos de valoración o ranking como el Elo rating, PageRank y pi-rating han sido herramientas recurrentes tanto en fútbol \cite{macleanReviewNFL20192022, owenDynamicBayesianForecasting2011, luizFootballUnpredictablePredicting2024} como en otros deportes como el baloncesto \cite{lampisPredictionsEuropeanBasketball2023}. Estos sistemas actualizan las valoraciones de los equipos tras cada partido, considerando la fortaleza del oponente y el resultado del encuentro.

    \item Con el objetivo de reflejar más fielmente el estado de forma actual del equipo, se han desarrollado extensiones. Por ejemplo, \cite{angeliniWeightedEloRating2022} introducen el Weighted Elo (WElo), que otorga mayor peso a los resultados recientes o a la contundencia del marcador, buscando así reflejar mejor el estado de forma actual, que es una manifestación temporal de la fortaleza del equipo. En esta misma línea, los modelos con "dynamic ratings" \cite{holmesForecastingFootballMatch2024} o los basados en el ranking Poisson para estimar parámetros de habilidad actual \cite{leyRankingSoccerTeams2019} también persiguen esta adaptación de calificación dinámica.
\end{itemize}

Si bien estos sistemas de calificación ofrecen una medida útil de la fortaleza global de un equipo, su carácter agregado inherentemente simplifica la realidad compleja del juego. Es decir, aunque nos dicen cuán fuerte es un equipo en general, no detallan en qué aspectos específicos reside esa fortaleza o dónde se encuentran sus vulnerabilidades concretas.

\subsubsection{Desglose del Rendimiento: Indicadores Clave (KPIs) para Detallar Fortalezas y Debilidades Específicas}

Para superar la limitación de los ratings globales y obtener una visión más granular, la investigación ha profundizado en el análisis de indicadores clave de rendimiento (KPIs) que reflejan facetas específicas del juego. Estos KPIs pueden interpretarse como manifestaciones observables de fortalezas o debilidades particulares de un equipo.

\begin{itemize}
    \item Las estadísticas descriptivas del partido como goles anotados, tiros totales y a puerta, precisión de pases, porcentaje de posesión, saques de esquina y faltas cometidas, son universalmente empleadas como primera capa para caracterizar el rendimiento y, por extensión, las capacidades de los equipos \cite{mPredictingOutcomeFuture2023, liPassingWinUsing2021, koningBettingMarketEfficiency2023}
    \item Trabajos como el de AlMulla et al. \cite{almullaSoccerNetGatedRecurrent2023} en la liga de Qatar, que analiza estadísticas técnico-físicas por intervalos temporales, o los que se enfocan en fuerzas ofensivas y defensivas enriquecidas con métricas como goles esperados (xG) \cite{koopmanForecastingFootballMatch2019,alvarezDataScienceApproach2024,wheatcroftForecastingFootballMatches2021}, buscan identificar las áreas donde los equipos demuestran superioridad o flaqueza.
    \item La identificación de KPIs esenciales mediante técnicas de reducción de dimensionalidad, como en el estudio de la Superliga de Malasia que se redujeron el número de KPIs con LDA y PCA de 38 a 8 \cite{abdulrahimDeterminationEssentialPerformance2023}, resalta las dimensiones del juego más críticas para el resultado. Del mismo modo, la influencia de las tarjetas en el desarrollo del juego también ha sido objeto de estudio \cite{badiellaInfluenceRedYellow2023}.
    \item La contribución de las características individuales de los jugadores es otro aspecto fundamental. Modelos basados en ratings individuales \cite{holmesForecastingFootballMatch2024, yeungFrameworkInterpretableMatch2023}, el impacto de ausencias clave \cite{zhangModelingPredictingOutcomes2021} o la evaluación de roles tácticos específicos, como el `falso 9', basándose en atributos individuales (ataque, técnica, creatividad, defensa) \cite{oluwayomiEvaluationTeamsFalse2022}, ilustran cómo las habilidades particulares de los jugadores se consideran fundamentales para definir las fortalezas y el rendimiento del equipo. La agregación de estas capacidades individuales permite construir una visión compuesta de las fortalezas y debilidades de una plantilla.
\end{itemize}

No obstante, a pesar de la granularidad que ofrecen los KPIs y el análisis a nivel de jugador, la literatura aún presenta oportunidades para una exploración más sistemática de cómo las fortalezas específicas de un equipo interactúan directamente con las debilidades particulares de su oponente. Muchos modelos incorporan estas métricas como factores independientes, sin modelar explícitamente el desajuste táctico o match-up específico. El presente proyecto busca abordar este área, con el objetivo de desarrollar un modelo que, al analizar estas interacciones detalladas junto con el contexto anímico, pueda identificar oportunidades de valor en el mercado de apuestas de la liga brasileña.
