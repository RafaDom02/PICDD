La predicción de resultados en el fútbol profesional ha empleado significativamente la evaluación de las capacidades de los equipos y jugadores, esto lo podemos ver utilizado desde papers hasta en aplicaciones móviles de resultados deportivos. Un componente central de esta evaluación es la identificación y cuantificación de sus fortalezas y debilidades, así como las de sus jugadores.

\subsubsection{Medición de la Fortaleza General del Equipo Mediante Sistemas de Calificación}

Una aproximación fundamental para medir la capacidad competitiva de los equipos se basa en sistemas de calificación (rating systems) que buscan condensar su fortaleza general en una métrica única o un conjunto reducido de ellas.

\begin{itemize}

    \item Métodos de valoración o ranking como el Elo rating, PageRank y pi-rating han sido herramientas recurrentes tanto en fútbol \cite{macleanReviewNFL20192022, owenDynamicBayesianForecasting2011, luizFootballUnpredictablePredicting2024} como en otros deportes como el baloncesto \cite{lampisPredictionsEuropeanBasketball2023}. Estos sistemas actualizan las valoraciones de los equipos tras cada partido, considerando la fortaleza del oponente y el resultado del encuentro.

    \item Con el objetivo de reflejar más fielmente el estado de forma actual del equipo, se han desarrollado extensiones. Por ejemplo, \cite{angeliniWeightedEloRating2022} introducen el Weighted Elo (WElo), que otorga mayor peso a los resultados recientes o a la contundencia del marcador, buscando así reflejar mejor el estado de forma actual, que es una manifestación temporal de la fortaleza del equipo. En esta misma línea, los modelos con "dynamic ratings" \cite{holmesForecastingFootballMatch2024} o los basados en el ranking Poisson para estimar parámetros de habilidad actual \cite{leyRankingSoccerTeams2019} también persiguen esta adaptación de calificación dinámica.
\end{itemize}

Si bien estos sistemas de calificación ofrecen una medida útil de la fortaleza global de un equipo, su carácter agregado inherentemente simplifica la realidad compleja del juego. Es decir, aunque nos dicen cuán fuerte es un equipo en general, no detallan en qué aspectos específicos reside esa fortaleza o dónde se encuentran sus vulnerabilidades concretas.

\subsubsection{Desglose del Rendimiento: Indicadores Clave (KPIs) para Detallar Fortalezas y Debilidades Específicas}

Para superar la limitación de los ratings globales y obtener una visión más granular, la investigación ha profundizado en el análisis de indicadores clave de rendimiento (KPIs) que reflejan facetas específicas del juego. Estos KPIs pueden interpretarse como manifestaciones observables de las fortalezas o debilidades particulares de un equipo, permitiendo un diagnóstico más preciso de su perfil competitivo.
\begin{itemize}
    \item Las estadísticas descriptivas del partido como goles anotados, tiros totales y a puerta, precisión de pases, porcentaje de posesión, saques de esquina y faltas cometidas, son universalmente empleadas como primera capa para caracterizar el rendimiento y, por extensión, las capacidades de los equipos \cite{mPredictingOutcomeFuture2023, liPassingWinUsing2021, koningBettingMarketEfficiency2023}. Por ejemplo, una alta tasa de conversión de tiros a puerta puede indicar una fortaleza en la finalización, mientras que un bajo porcentaje de pases completados en campo propio podría señalar una debilidad en la construcción inicial del juego.
    \item Un análisis más detallado del rendimiento se logra mediante la segmentación temporal o la consideración de métricas avanzadas. Trabajos como el de AlMulla et al. \cite{alMullaSoccerNetGatedRecurrent2023}, que examina estadísticas técnico-físicas por intervalos temporales en la liga de Qatar, permiten identificar si las fortalezas o debilidades de un equipo son consistentes a lo largo del partido o si emergen en fases críticas. De forma similar, el enfoque en fuerzas ofensivas y defensivas enriquecidas con métricas como los goles esperados (xG) y las asistencias esperadas (xA) \cite{koopmanForecastingFootballMatch2019, alvarezDataScienceApproach2024, wheatcroftForecastingFootballMatches2021}, busca cuantificar la calidad de las oportunidades generadas y concedidas, ofreciendo una medida más robusta de la fortaleza ofensiva y la solidez defensiva que los simples recuentos de goles. Un equipo con un xG consistentemente alto, incluso si no siempre se traduce en goles reales, demuestra una fortaleza en la creación de ocasiones.
    \item La identificación de los KPIs más influyentes a través de técnicas de reducción de dimensionalidad, como el Análisis de Componentes Principales (PCA) o el Análisis Discriminante Lineal (LDA), es crucial para aislar las dimensiones del juego que verdaderamente discriminan entre ganar, empatar o perder. El estudio de Abdulrahim et al. \cite{abdulrahimDeterminationEssentialPerformance2023} en la Superliga de Malasia, donde se redujo el conjunto de KPIs de 38 a 8, ilustra cómo identificar aquellos aspectos del rendimiento (fortalezas demostradas o debilidades evitadas) que son más críticos para el éxito. Paralelamente, la influencia de eventos disciplinarios como las tarjetas \cite{badiellaInfluenceRedYellow2023} también es relevante, la capacidad de un equipo para evitar expulsiones (manteniendo su fortaleza numérica) puede ser determinante.
    \item Finalmente, la contribución de las características individuales de los jugadores a las fortalezas y debilidades del colectivo es un aspecto fundamental. Modelos que se basan en ratings individuales de jugadores \cite{holmesForecastingFootballMatch2024, yeungFrameworkInterpretableMatch2023} reconocen que la calidad y el perfil de cada jugador son componentes esenciales del rendimiento global. El impacto de ausencias clave debido a lesiones o sanciones \cite{zhangModelingPredictingOutcomes2021} subraya cómo la no disponibilidad de un jugador con fortalezas particulares puede mermar significativamente la capacidad del equipo. Asimismo, la evaluación de roles tácticos específicos, como el `falso 9', basándose en atributos individuales (ataque, técnica, creatividad, defensa) \cite{oluwayomiEvaluationTeamsFalse2022}, ilustra cómo las habilidades particulares de los jugadores se consideran fundamentales para ejecutar ciertas estrategias y, por ende, para definir las fortalezas tácticas del equipo. La agregación y ponderación de estas capacidades individuales permite construir una visión más completa de las fortalezas y debilidades del equipo.
\end{itemize}
No obstante, a pesar de la granularidad que ofrecen los KPIs y el análisis a nivel de jugador, la literatura actual presenta una oportunidad significativa para una exploración más sistemática y profunda de cómo las fortalezas específicas de un equipo interactúan directamente con las debilidades particulares de su oponente. Si bien muchos modelos incorporan una variedad de estas métricas como factores independientes, a menudo no se modela explícitamente el desajuste táctico o \textit{match-up} específico que surge del enfrentamiento de perfiles concretos de fortalezas y debilidades. Este proyecto busca precisamente adentrarse en este espacio, con el objetivo de desarrollar un modelo que, al analizar estas interacciones detalladas junto con el contexto anímico y situacional obtenido mediante NLP, pueda identificar con mayor precisión oportunidades de valor en el mercado de apuestas de la liga brasileña, un entorno potencialmente menos explorado que las principales ligas europeas.
