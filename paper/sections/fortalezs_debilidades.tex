La predicción de resultados en el fútbol profesional ha empleado significativamente la evaluación de las capacidades de los equipos y jugadores, esto lo podemos ver utilizado desde papers hasta en aplicaciones móviles de resultados deportivos. Un componente central de esta evaluación es la identificación y cuantificación de sus fortalezas y debilidades, así como las de sus jugadores.

\subsubsection{Medición de la Fortaleza General del Equipo Mediante Sistemas de Calificación}

Una aproximación fundamental para medir la capacidad competitiva de los equipos se basa en sistemas de calificación (rating systems) que buscan condensar su fortaleza general en una métrica única o un conjunto reducido de ellas.

\begin{itemize}

    \item Métodos de valoración o ranking como el Elo rating, PageRank y pi-rating han sido herramientas recurrentes tanto en fútbol \cite{macleanReviewNFL20192022, owenDynamicBayesianForecasting2011} como en otros deportes como el baloncesto \cite{lampisPredictionsEuropeanBasketball2023}. Estos sistemas actualizan las valoraciones de los equipos tras cada partido, considerando la fortaleza del oponente y el resultado del encuentro.

    \item Con el objetivo de reflejar más fielmente el estado de forma actual del equipo, se han desarrollado extensiones. Por ejemplo, \cite{angeliniWeightedEloRating2022} introducen el Weighted Elo (WElo), que otorga mayor peso a los resultados recientes o a la contundencia del marcador, buscando así reflejar mejor el estado de forma actual, que es una manifestación temporal de la fortaleza del equipo. En esta misma línea, los modelos con "dynamic ratings" \cite{holmesForecastingFootballMatch2024} o los basados en el ranking Poisson para estimar parámetros de habilidad actual \cite{leyRankingSoccerTeams2019} también persiguen esta adaptación de calificación dinámica.
\end{itemize}

Si bien estos sistemas de calificación ofrecen una medida útil de la fortaleza global de un equipo, su carácter agregado inherentemente simplifica la realidad compleja del juego. Es decir, aunque nos dicen cuán fuerte es un equipo en general, no detallan en qué aspectos específicos reside esa fortaleza o dónde se encuentran sus vulnerabilidades concretas.
