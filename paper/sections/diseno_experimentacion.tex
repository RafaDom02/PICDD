\section{Diseño de la Experimentación}

Para evaluar la efectividad del modelo, se va a llevar a cabo una proceso de experimentación riguroso utilizando un amplio conjunto de datos históricos, que abarcan más de 10 temporadas del campeonato de fútbol brasileño. Las temporadas están comprendidas entre 2012 y 2023, formadas por cerca de 4000 partidos. 

La fase de pruebas del modelo se realizará con datos de partidos correspondientes a las temporadas 2024 y 2025, incluyendo tanto partidos ya finalizados como partidos en vivo. Se estima que serán utilizados cerca de 500 partidos.

Para evaluar el impacto de cada uno de los grupos de características del modelo, se van a realizar pruebas con modelos incrementales, lo que permitirá medir la contribución de cada tipo de información al rendimiento.

\begin{itemize}
    \item \textbf{Modelo base:} Compuesto únicamente por información con las categorías estándar y de rendimiento del equipo, definidas en la sección correspondiente a la definción de variables a inferir.
    \item \textbf{Modelo intermedio:} Compuesto por las características del modelo base, a las que se añaden las características contextuales.
    \item \textbf{Modelo completo:} Incluye las características del modelo intermedio y adicionalmente, incorpora las características avanzadas relativas a las fortalezas y debilidades de cada equipo.
\end{itemize}


El gran volumen de datos, tanto de entrenamiento como de pruebas, proporcionará una base sólida para poder construir modelos muy robustos que posibilitará obtener resultados significativos. 

La utilización de modelos incrementales no solo permitirá demostrar que el modelo completo desarrollado supera, en cuanto a ROI, a modelos previamente investigados, sino que también permitirá demostrar la importancia de cada uno de los grupos de características. De este modo, se podrá cuantificar el impacto de las variables contextuales y avanzadas en el rendimiento del sistema.