El número de investigaciones relacionadas con la predicción de resultados en eventos deportivos ha aumentado considerablemente en los últimos años, impulsado principalmente por el crecimiento de las casas de apuestas. En concreto, este fenómeno se ha visto reflejado en el aumento de estudios que buscan desarrollar modelos que no solo acierten el resultado de un evento, sino que además optimicen el retorno económico del usuario que realiza una apuesta. Este incremento en la actividad investigadora ha coincidido con un entorno donde las casas de apuestas han perfeccionado sus cuotas en las competiciones más populares, como las grandes ligas europeas, dificultando que los apostadores logren beneficios sostenidos en el tiempo.

Esta situación ha provocado que los nuevos trabajos busquen alternativas a los estudios más tradicionales, buscando obtener una mayor rentabilidad, analizando ligas menos estudiadas y con menor volumen de apuestas, donde la eficiencia de las casas de apuestas es significativamente inferior. En este contexto, este trabajo propone una investigación orientada a conseguir un modelo de predicción que maximiza el retorno de inversión (ROI) de un usuario al realizar una apuesta, a través de la identificación de cuotas competitivas, y, por tanto, desajustadas en relación a la previsión real del resultado, en competiciones de segundo nivel. En concreto, se centra en la liga brasileña de fútbol, una competición que en los últimos años está teniendo una pérdida de interés comercial, pero que sigue siendo una liga con un gran potencial desde el punto de vista analítico ya que se disponen de datos suficientes para su estudio.

A diferencia de otros trabajos que únicamente predicen el resultado final de un partido, en este trabajo se desarrolla un modelo que tiene como objetivo principal encontrar oportunidades de apuesta que resulten rentables, combinando técnicas de aprendizaje automático con el análisis del contexto emocional que rodea a los equipos. Para ello, se utilizan fuentes de información más estudidas en la literatura, como pueden ser las estadísticas de rendimiento y resultados anteriores, junto con datos extraídos de redes sociales y medios deportivos, lo que permite utilizar variables relacionadas con el sentimiento de los aficionados, la moral del equipo o la posible ausencia de jugadores clave.

El modelo trabaja con datos históricos de partidos, cuotas, características físicas y técnicas de los equipos y jugadores y variables contextuales como la sintonía táctica o el ambiente emocional previo al partido. Estas características se combinan a través de un modelo de aprendizaje automático con el objetivo de realizar una predicción de la probabilidad de victoria local, empate o victoria visitante, combinando dichas probabilidades con las cuotas ofrecidas por las casas de apuestas para optimizar el ROI del usuario. En este estudio se van a utilizar tres tipos de modelos de \textit{machine learning}: modelos basados en \textit{gradient boosting}, \textit{random forest} y redes neuronales profundas, todos ellos contrastados ampliamente en la literatura científica.

Para entrenar el modelo, se utilizarán datos de más de 5.000 partidos disputados en la liga brasileña entre 2012 y 2025, junto con un conjunto de datos de cuotas pre-partido, características de los jugadores y un análisis de las noticias de los diferentes equipos en el periodo temporal anterior al partido. Además, se llevará a cabo un diseño incremental que permitirá observar y cuantificar el uso de cada grupo de variables (básicas, contextuales y avanzadas) sobre el rendimiento del modelo, mostrando cómo la incorporación del análisis de sentimientos disminuye los errores en términos de rentabilidad.

En definitiva, este trabajo plantea una herramienta útil tanto para apostadores que buscan mejorar su beneficio económico como para casas de apuestas interesadas en detectar cuotas mal ajustadas. De esta manera, se presenta una propuesta innovadora que contribuye de forma clara al campo de la predicción deportiva, ofreciendo un modelo robusto, justificado y alineado con el objetivo de maximizar el rendimiento en mercados con menor eficiencia.




