La predicción de resultados en eventos deportivos ha ganado tracción investigadora, impulsada por la industria de las apuestas. Sin embargo, la creciente eficiencia de las casas de apuestas en competiciones principales limita la rentabilidad. Este estudio aborda este desafío centrándose en ligas menos populares, como la liga brasileña de fútbol, donde la eficiencia del mercado es menor y existen mayores oportunidades. El objetivo principal es desarrollar un modelo de predicción que no solo acierte resultados, sino que maximice el retorno de inversión (ROI) del apostador mediante la identificación de cuotas infravaloradas.

Para ello, se propone un enfoque innovador que integra técnicas de aprendizaje automático (incluyendo gradient boosting, random forest y redes neuronales profundas) con el análisis del contexto emocional que rodea a los equipos. Se utilizarán datos históricos de más de 5.000 partidos (2012-2025), cuotas pre-partido, estadísticas de rendimiento, y, crucialmente, datos extraídos de redes sociales y medios deportivos para capturar variables como el sentimiento de los aficionados, la moral del equipo y la ausencia de jugadores clave.

El modelo combinará estas diversas fuentes de información para predecir las probabilidades de victoria local, empate o victoria visitante, las cuales se contrastarán con las cuotas ofrecidas para optimizar las decisiones de apuesta. Un diseño experimental incremental permitirá cuantificar el valor añadido de cada grupo de variables, con especial énfasis en cómo la incorporación del análisis de sentimientos contribuye a mejorar la rentabilidad del modelo. En definitiva, este trabajo busca ofrecer una herramienta robusta y justificada tanto para apostadores como para casas de apuestas interesadas en mercados menos eficientes, aportando una contribución significativa al campo de la predicción deportiva.


