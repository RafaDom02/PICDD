\section{Diseño de la Experimentación}

Para evaluar la efectividad del modelo se va a llevar a cabo una proceso de experimentación rigurosa, en el cual se realizará un estudio con un gran conjunto de datos históricos, conformados por más de 10 temporadas del campeonato de fútbol brasileño, comprendidas desde 2012 hasta 2023, que suponen cerca de 4000 partidos. Para llevar a cabo las pruebas del modelo se pretende utilizar partidos de los años 2024 y 2025, incluyendo partidos ya jugados y partidos en vivo, siendo utilizados cerca de 500 partidos. 

Para evaluar la eficiencia de todas las características del modelo, se van a realizar pruebas con modelos incrementales, partiendo de un modelo básico que cuenta con un conjunto de características esenciales, al cual se le irán incorporando nuevas características hasta alcanzar el modelo más robusto.

\begin{itemize}
    \item Modelo compuesto únicamente por información con las categorías, definidas durante el apartado de dataset, estándar y de rendimiento del equipo.
    \item Modelo compuesto por las características del modelo anterior y añadiendo las características contextuales.
    \item Modelo que a parte de todas las características de los modelo previos contará con información sobre las fortalezas y debilidades de cada uno de los equipos.
\end{itemize}

Se considera que gracias al gran volumen de datos, tanto de entrenamiento como de pruebas, nos permitirá desarrollar modelos muy robustos que permitirán obtener resultados significativos. 

Por la utilización de modelos incrementales no solo permitirá demostrar que el modelo desarrollado es mejor, en cuanto a ROI, a modelos ya experimentados y disponibles en múltiples investigaciones, además será posible demostrar que cada uno de los grupos de características escogido supone una mejora sustancial en cuanto a los resultados de estudios previos a medida que se va incrementando el volumen de características que se están utilizando.