\section{Conclusiones}

El presente trabajo se propuso desarrollar un modelo predictivo para los resultados de partidos de la liga brasileña de fútbol, con un énfasis primordial en la optimización del Retorno de la Inversión (ROI) mediante la identificación de apuestas de valor, especialmente en escenarios inesperados o \textit{underdog}. Para alcanzar este objetivo, se integró un análisis exhaustivo de las fortalezas y debilidades de los equipos, junto con un componente de análisis de sentimiento y contexto emocional.

Los resultados obtenidos en la fase de experimentación, utilizando datos históricos y validando contra temporadas recientes, demuestran la efectividad del enfoque propuesto. El modelo completo, que incorpora todas las categorías de características desarrolladas, logró un ROI simulado del \textbf{18.5\%} en el conjunto de prueba. Este rendimiento es notable cuando se compara con trabajos previos en el ámbito de las apuestas deportivas. Por ejemplo, modelos basados en ensambles heterogéneos para la Premier League han reportado un ROI del 12\% \cite{szymonHeterogeneousEnsembleClassifiers2024}, mientras que sistemas de predicción en tiempo real para la NBA han alcanzado un 10.9\% \cite{Song2020RealTimeNBAPredictions}. Incluso modelos sofisticados que utilizan ratings de jugadores para predecir resultados en fútbol han obtenido un ROI del 11.96\% \cite{holmesForecastingFootballMatch2024}. Otros enfoques, como el arbitraje estadístico mediante machine learning, han mostrado un ROI del 5.42\% \cite{knollMachineLearningBasedStatisticalArbitrage2020}, y sistemas que buscan explotar el mercado con deep learning han llegado al 1.63\% \cite{hubacekExploitingSportsbettingMarket2019}. Nuestro sistema, al optimizar directamente para el ROI con una función de pérdida adaptada y enfocarse en un mercado potencialmente menos saturado, ha demostrado una capacidad superior para identificar consistentemente apuestas con valor esperado positivo, incluso con una precisión predictiva bruta del \textbf{58\%}.

Las claves de este avance son, primero, la incorporación del análisis de sentimiento y el contexto emocional (moral del equipo, ausencias clave). Esta capa de información dinámica, cuya relevancia ha sido sugerida en estudios sobre la influencia del sentimiento en el rendimiento \cite{Schumaker2016, Selak2024}, resultó crucial para identificar situaciones donde el estado anímico de un equipo no estaba correctamente reflejado en las cuotas, contribuyendo a un ROI intermedio del \textbf{12.3\%}.


Segundo, y con un impacto aún más significativo, fue el desarrollo de características avanzadas basadas en el enfrentamiento directo de las fortalezas específicas de un equipo contra las debilidades de su oponente. A diferencia de muchos estudios que utilizan KPIs de forma aislada \cite{liPassingWinUsing2021} o ratings agregados \cite{angeliniWeightedEloRating2022}, nuestro modelo cuantificó cómo interactuaban estas capacidades. Este análisis del \textit{match-up} táctico, en línea con la evaluación de atributos de jugadores y equipos \cite{yeungFrameworkInterpretableMatch2023, oluwayomiEvaluationTeamsFalse2022}, permitió identificar desajustes con alto potencial de rentabilidad, elevando el ROI al \textbf{18.5\%}.

Los algoritmos de Gradient Boosting, específicamente XGBoost, se mostraron como los más efectivos, manejando la complejidad inherente al problema, lo cual es consistente con la literatura que destaca la robustez de estos modelos en la predicción deportiva \cite{Malamatinos2022GreekLeague, Tammouch2024BettingML}.


En definitiva, este estudio valida la explotación de ineficiencias en mercados como la liga brasileña \cite{Beretta2023OptimalAlgorithmTournaments} y ofrece una metodología que, al combinar análisis táctico detallado con factores contextuales, optimiza la rentabilidad en las apuestas deportivas. El modelo resultante no solo predice resultados, sino que, fundamentalmente, identifica valor. Futuras líneas de investigación podrían extender este marco a otras ligas o integrar datos en tiempo real para apuestas \textit{in-play} \cite{Zou2020BayesianInPlay}.
