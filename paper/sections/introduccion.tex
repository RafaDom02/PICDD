\section{Introducción}

El número de apuestas deportivas se ha incrementado considerablemente en los últimos años, concretamente, en 2024 se registró un crecimiento del 23,80\% en comparación con 2023, alcanzando ingresos de 608,85 millones de euros \cite{casinos2024}. Este crecimiento se mantendrá los próximos años, llegando a alcanzar una facturación de 153.71 billones de dólares en 2029, aplicando una tasa de crecimiento compuesto del 5.4\% de forma anual \cite{tbrc2025sportsbetting}.

El aumento en la popularidad de las apuestas deportivas ha provocado también un aumento significativo en el número de investigaciones académicas sobre este tema. Actualmente, muchos estudios se han centrado en aplicar diferentes técnicas estadísticas y modelos de aprendizaje automático con el objetivo de predecir correctamente los resultados de eventos deportivos. Se puede comprobar al hacer una búsqueda en IEE, filtrando por años de publicación, como en 2024 el número de \textit{papers} publicado en este área creció considerablemente. Esto se debe principalmente al interés de los usuarios por encontrar métodos que les permitan obtener un beneficio económico aplicando conocimientos teóricos justificados y razonados.

Sin embargo, las casas de apuestas deportivas buscan garantizar su rentabilidad y su crecimiento económico, por ello, invierten grandes cantidades de dinero en asegurar que sus cuotas son lo suficientemente competitivas para atraer a los apostadores, pero no lo suficientemente altas como para que los apostadores obtengan beneficios consistentes. Para ello, principalmente buscan asegurar sus ingresos en los eventos más populares y con mayor volumen de apuestas, como las grandes ligas europeas. Debido al gran número de apuestas en estas competiciones, las casas de apuestas analizan de forma muy detallada estos partidos, lo que dificulta enormemente que un usuario promedio logre obtener beneficios de forma constante.

En este contexto, este trabajo plantea una estrategia alternativa para mejorar el rendimiento económico de los usuarios enfocándose en ligas menos populares. Son en estos partidos donde las casas de apuestas al no registrar tantos usuarios, no dedican recursos al estudio y el análisis de las cuotas. En las grandes ligas, al contar con un mayor número de apuestas, las cuotas están muy estudiadas y ajustadas, ya que un error o un valor ligeramente desviado, debido al gran número de apuestas que se realizan, podría suponer un pérdida importante de beneficios. Aunque esta estrategia presenta la desventaja adicional de tener una menor disponibilidad de datos históricos para desarrollar modelos predictivos precisos, también tiene la ventaja de enfrentar a equipos menos estudiados, y por tanto, con cuotas menos ajustadas, lo que ofrece una oportunidad clara para obtener mayores beneficios. 

En concreto, este estudio se centra en la liga brasileña de fútbol, una competición que actualmente está perdiendo popularidad tanto en el ámbito nacional como en el internacional. Según un estudio reciente citado en \cite{sportsvalue2025}, el mercado de fútbol en Brasil es uno de los más grandes del planeta, pero está absolutamente subaprovechado por parte de los clubes de Brasil, creciendo cada año el desinterés por las marcas de los equipos brasileños, y aumentando el interés por los equipos europeos.

En la literatura se han encontrado numerosos trabajos académicos aplicados a las apuestas deportivas, pero la mayoría de ellos centrados únicamente en predecir el resultado final del partido, sin embargo, en este trabajo se va a utilizar como métrica el retorno de inversión (ROI). De esta forma, se busca obtener un modelo que optimice sus predicciones en función de las ganancias, consiguiendo así un modelo que busca maximizar el beneficio del usuario, y no únicamente predecir el resultado de un partido.

Como datos de entrada del modelo, se van a utilizar: las características clásicas presentes en la literatura aplicadas para cada equipo, datos comparativos entre los equipos que se enfrentan y como método de innovación, se incluyen variables relacionadas con el contexto actual de los equipos, analizando sus sentimiento en relación a la situación actual en la liga, el ambiente de la afición mediante el estudio en redes sociales, posibles lesiones de jugadores clave en el club o la sintonía entre jugadores. Estas características proporcionarán una nueva visión de los equipos, que se reflejará en mejores predicciones.

Para desarrollar y evaluar el modelo, se van a probar varios algoritmos de aprendizaje automático, concretamente se van a utilizar modelos basados en \textit{gradient boosting}, \textit{random forest} y redes neuronales.

Finalmente, optimizando el ROI, se va a conseguir un modelo que indique la probabilidad de que el equipo local gane, empate o pierda. Obteniéndose una mayor probabilidad teniendo en cuenta tanto el resultado del partido como la cuota a la que está fijada el partido por la casa de apuestas.


