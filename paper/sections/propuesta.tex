\section{Propuesta}

El **nombre proyecto** tiene como objetivo el desarrollo de un modelo que combine la predicción de resultados deportivos con el descubrimiento de posibles sorpresas que supongan una alta rentabilidad en el mercado de apuestas. Para lograrlo, se llevará a cabo un análisis estadístico exhaustivo que tendrá en cuenta tanto debilidades como fortalezas y el contexto de cada uno de los equipos.

Este modelo pretende optimizar el retorno de inversión (ROI) en un mercado de apuestas menos eficiente respecto a las grandes ligas europeas, como es la liga brasileña de fútbol.

Los principales objetivos del proyecto son:

\begin{enumerate}
    \item Construir un conjunto de estadísticas avanzadas que permitan la comparación de las debilidades y fortalezas, a partir de las características tanto a colectivas como individuales, destacando a aquellos jugadores que van a participarán de inicio en el encuentro.
    \item Incorporar análisis de sentimiento a partir de noticias y opiniones en redes sociales por parte de aficionados y medios especializados, para tener información contextual del estado de forma de los equipos, resultados en partidos previos y disponibilidad de jugadores.
    \item Construir un modelo de aprendizaje automático capaz de predecir los resultados de los encuentros de manera eficiente y aprovechar las características construidas para maximizar el ROI en las apuestas a realizar.
\end{enumerate}

El proyecto está destinado a apostadores que quieren incrementar su rendimiento, y por tanto beneficios, a la hora de realizar apuestas gracias a las mejoras en la predicción y, especialmente, del retorno de inversión (ROI). Por otro lado, también tiene el potencial para ser utilizado por parte de las casas de apuestas, dado que les permitirá poder encontrar posibles cuotas mal calibradas, y por tanto, susceptibles de producir pérdidas no esperadas, especialmente en mercados minoritarios.

\subsection{Datasets a utilizar}

Para poder construir el sistema planteado será necesario la utilización de múltiples fuentes de datos que cubran los distintos aspectos del análisis.

\begin{enumerate}
    \item \textbf{Datos históricos de partidos:} Se incluyen los resultados de los partidos de la liga brasileña de fútbol a lo largo de distintas temporadas. Esta datos se pueden encontrar en el portal football data \cite{footballData}, que ofrece resultados de más de 5000 partidos desde la temporada 2012 hasta 2025.

    \item \textbf{Cuotas de apuestas:} Para afinar el retorno de inversión (ROI), es necesario contar con información de las cuotas previas a los partidos. Esta información se puede encontrar en el portal nowgoal \cite{nowgoalOdds}, donde se disponen de cuotas de partidos de la liga brasileña desde 2005 hasta 2025 con más de 30 casa de apuestas distintas.
    
    \item \textbf{Características de los jugadores y equipos:} Se emplerán datos con información detallada relativos a características técnicas y físicas tanto de equipos como de jugadores. Estas características permitirán definir las fortalezas y debilidades para ajustar los enfrentamientos. La información se encuentra disponible en los portales datamb \cite{datamb} y statsbomb \cite{statsbomb}.
    \item \textbf{Información contextual:} Con el objetivo de obtener información del estado de los equipos, se analizarán noticas de medios especializados y opiniones en redes sociales previas a los encuentros, especialmente del portal X \cite{twitter}, mediante el uso de análisis de sentimientos. Adicionalmente, se completará esta información con detalles sobre alineaciones, sanciones y lesiones disponibles plataformas como sofascore \cite{sofascore}.
\end{enumerate}

En la literatura se encuentra la utilización de datos históricos, como en el trabajo Holmes et al. \cite{holmesForecastingFootballMatch2024}, cuotas de apuestas, como en el trabajo Fischer et al. \cite{Fisher2024PricingResponse} y el uso de redes sociales para obtener información contextual, como en el trabajo de Schumaker et al. \cite{Schumaker2016}. Sin embargo, la utilización de conjuntos de datos que cuentan con características avanzadas de equipos y jugadores no se encuentran tan explorados en la actualidad.

La información relativa a cuotas de casas de apuestas, características de jugadores y equipos e información contextual no están disponible de manera estructurada en un conjunto de datos, por lo que será necesario desarrollar una serie de pipelines de recolección de dichos datos mediante procesos de web scraping. 

Asimismo, la diversidad y heterogeneidad de las fuentes de datos empleadas implica la necesidad del desarrollo de un arduo proceso de limpieza y transformación de datos previo, con el fin de construir un sistema robusto, confiable y eficiente. 

\subsection{atributos a inferir}


Los principales atributos de entrada del sistema se dividen en las siguientes categorías:

\begin{enumerate}
    \item \textbf{Características estándar:}
        \begin{itemize}
            \item Cuotas prepartido: Valores ofrecidos por las casas de apuestas respecto a la victoria local, victoria visitante o el empate.
            \item Ventaja de campo: Indica cuál de los equipos está jugando en su estadio, lo cual puede tener gran influencia en los resultados.
        \end{itemize}
    \item Rendimiento del equipo: Incluyendo información tanto a nivel global como específico, considerando enfrentamientos previos entre equipos y diferenciando entre encuentros como local y visitante.
    \begin{itemize}
        \item Estado de forma: Medida que indica el rendimiento en los partidos previos, dando mayor importancia a los partidos más recientes.
        \item Momentum: indica la racha que tiene el equipo, ya sea de victorias, derrotas o empates, y su duración.
        \item Fortaleza del equipo: Indica que tan fuerte es un equipo relativo a un sistema de ELO dinámico.
    \end{itemize}
    \item \textbf{Características contextuales:}
    \begin{itemize}
        \item Puntuación del sentimiento prepartido: Valor derivado del análisis de sentimiento de noticias publicadas por medios especializados y opiniones en redes sociales.
        \item Jugadores disponibles: Presencia o ausencia de jugadores debido a lesiones, sanciones u otras causas.
        \item Jugadores titulares: Jugadores que van a comenzar el partido, o se espera que lo hagan, y por tanto, van a tener una gran influencia en el resultado.
        \item Tiempo de descanso relativo: El tiempo de descanso respecto a partidos previos ponderando la fortaleza de los equipos a los que se ha enfrentado.
    \end{itemize}
    \item \textbf{Características avanzadas:} Reflejan las fortalezas y debilidades de cada uno de los equipos, construidas a partir de las características generales del equipo unido a las características individuales de los jugadores que van a ser, o se predice que van a ser, titulares.
    \begin{itemize}
        \item Goles esperados: Goles esperados a favor y en contra.
        \item Presión: Capacidad de realizar presión efectiva cuando no se cuenta con la posesión del balón.
        \item Precisión de pases bajo presión: Habilidad a la hora de realizar pases en momentos comprometidos.
        \item Duelos aéreos: Efectividad en disputas aéreas, tanto ofensivas como defensivas.
        \item Duelos terrestres: Efectividad en disputas terrestres, tanto ofensivas como defensivas.
        \item Habilidad de contraataque: Capacidad de generar o detener situaciones de peligro en transiciones.
         \item Habilidad juego posesión: capacidad de generar o detener situaciones de peligro en situaciones de peligro generadas por posesiones largas.
    \end{itemize}
\end{enumerate}

En la literatura existente se puede encontrar la utilización de cuotas prepartido y relevancia de disponibilidad de jugadores, como en el trabajo Fischer et al. \cite{Fisher2024PricingResponse}, puntuación de análisis de sentimiento, como en el trabajo Schumaker et al. \cite{Schumaker2016}, ventaja de campo, como en el trabajo Fontanella et al. \cite{Fontanella2020Visual} y fortaleza del equipo, como en el trabajo Ley et al. \cite{leyRankingSoccerTeams2019}. Por su parte, todas las características avanzadas que han sido construidas no se encuentran muy exploradas actualmente.

Será necesario un arduo proceso de preprocesamiento de los datos, especialmente, por la necesidad de llevar a cabo un análisis de sentimiento para poder construir la puntuación de sentimiento prepartido, así como de la construcción de las características avanzadas, ya que dependen de los jugadores que van a ser partícipes desde el principio del encuentro.

El sistema contará como salida con las probabilidades de obtener cada uno de los tres posibles resultados de un partido:
\begin{itemize}
    \item Probabilidad de victoria del equipo local.
    \item Probabilidad de empate.
    \item Probabilidad de victoria del equipo visitante.
\end{itemize}

Estas probabilidades serán combinadas con las cuotas de casas de apuestas con el fin de detectar oportunidades que permitan maximizar el retorno de inversión (ROI).

\subsection{Métodos a utilizar}

Dada la complejidad del problema y la naturaleza no lineal de muchos de los atributos, se ha considerado que lo más apropiado es utilizar diferentes modelos de machine learning robustos. Sobre ellos se llevarán a cabo comparaciones de su desempeño para determinar cuál es el modelo que maximiza el retorno de inversión (ROI). Los modelos que han sido seleccionados son los siguientes:

\begin{itemize}
    \item \textbf{Modelos basados en Gradient Boosting como pueden ser XGboost, LightGMB o CatBoost:} Se tratan de modelos basados en conjuntos, que son modelos muy robustos que presentan buenos rendimientos trabajando con problemas con relaciones no linealeas y de gran dimensionalidad.
    \item \textbf{Rodom Forest:} Modelo de conjuntos de árboles de decisión, también muy robusto, que permite tratar con grandes cantidades de características de entrada.
    \item \textbf{Red Neuronal:} Se utilizarán redes neuronales profundas, por su capacidad para tratar con conjuntos de datos con características de gran complejidad, incluso si cuentan con relaciones altamente no lineales.
\end{itemize}

Los tres modelos seleccionados son ampliamente utilizados en la literatura actual para la predicción de resultados deportivos. En el trabajo Malamatinos et al. \cite{Malamatinos2022GreekLeague} podemos encontrar el uso de Random Forest y Gradient Boosting, mientras que en el trabajo Tammouch et al. \cite{Tammouch2024BettingML} se utilizan redes neuronales, especialmente centrado en redes neuronales profundas.

Como el objetivo principal del proyecto no es solo predecir el resultado de cada uno de los encuentros, sino optimizar el ROI, será necesario crear de una función de pérdida que tenga en cuenta las cuotas de apuesta de cada uno de los partidos. De esta manera, el sistema no solo permitirá predecir con suficiencia el resultado del partido, sino que también podrá identificar partidos con un alto potencial de rentabilidad.

Estos resultados con gran potencial se pueden asociar a resultados menos esperados. Su identificación será posible gracias a las variables contextuales introducidas en el proyecto, tales como el estado de forma, el análisis de sentimiento respecto a cada uno de los equipos, y, por otro lado, por el descubrimiento de enfrentamientos favorables al equipo a priori más débil, gracias a la identificación de las fortalezas y debilidades.